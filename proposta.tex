\documentclass{article}
\usepackage[utf8]{inputenc}
\usepackage{amsmath, amssymb}
\usepackage[margin=1in]{geometry}

\title{Portfolio Optimization}
\author{Lucas Ximenes Guilhon}

\begin{document}

\maketitle

Portfolio Optimization is a problem that can take on many different forms. We will establish what a general deterministic portfolio optimization problem looks like, and then
propose improvements to it, using the tools that we've learned and will learn in the course.
\section*{Formulation}
Our base problem will consider the following properties/parameters:
\begin{itemize}
\item $M$ assets, and $r_n$ is the expected return of the $n$-th asset.
\item The amount allocated to each asset $w_1, w_2, \dots, w_M$ must sum up to our total budget $B$.
\item We are strictly buying assets, so $w_i \geq 0$ for all $i$.
\item We wish to maximize the expected return of the portfolio.
\end{itemize}
The deterministic form of the problem described above is given by:
\begin{equation}
  \begin{aligned}
  \max_{w_1, w_2, \dots, w_M} & \sum_{n=1}^M w_n r_n\\
  \textrm{s.t.} \quad & \sum_{n=1}^M w_n = B\\
    & w_n \geq 0 \quad \forall n=1..M
  \end{aligned}
\end{equation}
\section*{Improvements}
The main improvements that we will propose are:
\begin{itemize}
  \item Introduce robustness to the problem by considering a distribution of possible returns for each asset. $r_i$ now becomes $\bar{r}_i \in R_i$ where $R_i$ is the set of possible returns for asset $i$.
  \item Utilize a risk measure (preferably a coherent one, like CV@R) and constrain our portfolio to be within a certain risk tolerance $T$.
\end{itemize}
But it's worth mentioning that other improvements can be added as the project progresses. A short consideration to be made, is that, considering second improvement mentioned above,
we could model our optimization problem as a minimization of the risk constrained by minimal expected return. This is one of the many ways to interoperate risk and return in a portfolio optimization problem
and we may explore one or more of these possibilities.
\end{document}
