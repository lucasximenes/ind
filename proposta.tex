\documentclass{article}
\usepackage[utf8]{inputenc}
\usepackage{amsmath, amssymb}
\usepackage[margin=1.25in]{geometry}

\title{Portfolio Optimization}
\author{Lucas Ximenes Guilhon}

\begin{document}

\maketitle

Portfolio Optimization is a problem that can take on many different forms. We will establish what a general deterministic portfolio optimization problem looks like, and then
propose improvements to it, using the tools that we've learned and will learn in the course.
\section*{Formulation}
Our base problem will consider the following properties/parameters:
\begin{itemize}
\item $M$ assets, and $r_n$ is the expected return of the $n$-th asset.
\item The amount allocated to each asset $w_1, w_2, \dots, w_M$ must sum up to our total budget $B$.
\item We are strictly buying assets, so $w_i \geq 0$ for all $i$.
\item We wish to maximize the expected return of the portfolio.
\end{itemize}
The deterministic form of the problem described above is given by:
\begin{equation}
  \begin{aligned}
  \max_{w_1, w_2, \dots, w_M} & \sum_{n=1}^M w_n r_n\\
  \textrm{s.t.} \quad & \sum_{n=1}^M w_n = B\\
    & w_n \geq 0 \quad \forall n=1..M
  \end{aligned}
\end{equation}
\section*{Improvements}
The main improvements (other improvements can be added as the project progresses) that we will propose are:
\begin{itemize}
  \item Introduce robustness to the problem by considering a distribution of possible returns for each asset. $r_i$ now becomes $\bar{r}_i \in R_i$ where $R_i$ is the set of possible returns for asset $i$.
  \item Utilize a risk measure (preferably a coherent one, like CV@R) and constrain our portfolio to be within a certain risk tolerance $T$.
\end{itemize}
A short consideration to be made, is that, considering second improvement mentioned above,
we could model our optimization problem as a minimization of the risk constrained by minimal expected return. This is one of the many ways to inter operate risk and return in a portfolio optimization problem
and we may explore one or more of these possibilities.
\section*{Development}
\begin{equation}
  \begin{aligned}
  \max_{x \in R^n} &\quad \bar{r}^\intercal x\\
  \textrm{s.t.} \quad & \sum_{i=1}^M x_i = B\\
    & x_i \geq 0 \quad \forall i=1..M
  \end{aligned}
\end{equation}
Adding a risk constraint.
\begin{equation}
  \begin{aligned}
  \max_{x \in R^n} &\quad \bar{r}^\intercal x\\
  \textrm{s.t.} \quad & \sum_{i=1}^M x_i = 1\\
    & x_i \geq 0 \quad \forall i=1..M\\
    & CV@R(x) \leq T_r
  \end{aligned}
\end{equation}
Thankfully, the CV@R can be formulated as a linear programming problem, so we can use it as a constraint in our optimization.
The initial, non-linear formulation, is given by:
\begin{equation}
  CV@R(X) = \min_{l \geq 0} \quad l + \frac{1}{1-\alpha} \sum_{i=1}^M p_i \cdot \max\{0, X - l\}
\end{equation}
And in order for it to become a linear programming problem, we need to reformulate the maximization with the following constraints:
\begin{equation}
  \begin{aligned}
    CV@R(X) = \min_{l \geq 0} & \quad l + \frac{1}{1-\alpha} \sum_{i=1}^M p_i \cdot \theta_i\\
    \textrm{s.t.} \quad & \theta_i \geq X_i - l \quad \forall i=1..M\\
    & \theta_i \geq 0 \quad \forall i=1..M
  \end{aligned}
\end{equation}
Where X is, in our case, the set of all possible profits and losses, $p_i$ is the probability of each return, and $\alpha$ is the confidence level.
Now, we can substitute the CV@R constraint in our original problem, and we get:
\begin{equation}
  \begin{aligned}
  \max_{x, l, \theta} &\quad \bar{r}^\intercal x\\
  \textrm{s.t.} \quad & \sum_{i=1}^M x_i = 1\\
    & x_i \geq 0 \quad \forall i=1..M\\
    & \min_{l \geq 0} \quad l + \frac{1}{1-\alpha} \sum_{i=1}^M p_i \cdot \theta_i \leq T_r\\
    & \theta_s \geq 0, \quad \forall s=1..N\\
    & \theta_s \geq -r_sx - l, \quad \forall s=1..N\\
  \end{aligned}
\end{equation}
However, we've now introduced a minimization problem inside a constraint 
\begin{equation}
  \begin{aligned}
  \max_{x, l, \theta} &\quad \bar{r}^\intercal x\\
  \textrm{s.t.} \quad & \sum_{i=1}^M x_i = 1\\
    & x_i \geq 0 \quad \forall i=1..M\\
    & l + \frac{\sum_s p_s \theta_s}{1-\alpha} \leq T_r\\
    & \theta_s \geq 0, \quad \forall s=1..N\\
    & \theta_s \geq -r_sx - l, \quad \forall s=1..N\\
  \end{aligned}
\end{equation}
In order to add the robust constraint that enforces the asset allocation to have a greater return than a specific threshold. The primal version of this constraint is the following problem:
\begin{equation}
  \begin{aligned}
  \min_{r, z\in R^n} &\quad \bar{r}^\intercal x\\
  \textrm{s.t.} \quad
    & z_i\sigma_i - r_i \geq -\bar{r}_i\\
    & z_i\sigma_i + r_i \geq \bar{r}_i\\
    & -z_i \geq -1\\
    & -\sum_i z_i \geq - \Gamma\\
  \end{aligned}
\end{equation}
And the dual is the following:
\begin{equation}
  \begin{aligned}
  \max_{\lambda, \beta, \Delta^+, \Delta^- \leq 0} &\quad -\Gamma \beta - \sum_i \lambda_i + \sum_i \bar{r}_i(\Delta_i^+ - \Delta_i^-) \\
  \textrm{s.t.} \quad & \Delta_i^+ - \Delta_i^- \geq x_i\\
    & \sigma_i(\Delta_i^+ - \Delta_i^-) - \lambda_i - \beta \geq 0\\
  \end{aligned}
\end{equation}
Now, we can obtain our problem with robust and risk-aversion constraints:
\begin{equation}
  \begin{aligned}
  \max_{x, l, \theta, \lambda, \beta, \Delta^+, \Delta^-, T_p} &\quad T_p\\
  \textrm{s.t.} & \quad T_p \leq \bar{r}_i\\
    & x_i \geq 0 \quad \\
    & l + \frac{\sum_s p_s \theta_s}{1-\alpha} \leq T_r\\
    & x_i = x_{t-1} + c_i - v_i - \gamma(c_i + v_i)\\
    & \theta_s \geq 0, \quad \forall s=1..N\\
    & \theta_s \geq -r_sx - l, \quad \forall s=1..N\\
    &-\Gamma \beta - \sum_i \lambda_i + \sum_i \bar{r}_i(\Delta_i^+ - \Delta_i^-) \geq T_p\\
    & \Delta_i^+ - \Delta_i^- \geq x_i\\
    & \sigma_i(\Delta_i^+ - \Delta_i^-) - \lambda_i - \beta \geq 0\\
  \end{aligned}
\end{equation}
\end{document}
